% THIS IS SIGPROC-SP.TEX - VERSION 3.0
% WORKS WITH V3.1SP OF ACM_PROC_ARTICLE-SP.CLS
% JUNE 2007
%
% It is an example file showing how to use the 'acm_proc_article-sp.cls' V3.1SP
% LaTeX2e document class file for Conference Proceedings submissions.
% ----------------------------------------------------------------------------------------------------------------
% This .tex file (and associated .cls V3.1SP) *DOES NOT* produce:
%       1) The Permission Statement
%       2) The Conference (location) Info information
%       3) The Copyright Line with ACM data
%       4) Page numbering
% ---------------------------------------------------------------------------------------------------------------
% It is an example which *does* use the .bib file (from which the .bbl file
% is produced).
% REMEMBER HOWEVER: After having produced the .bbl file,
% and prior to final submission,
% you need to 'insert'  your .bbl file into your source .tex file so as to provide
% ONE 'self-contained' source file.
%
% Questions regarding SIGS should be sent to
% Adrienne Griscti ---> griscti@acm.org
%
% Questions/suggestions regarding the guidelines, .tex and .cls files, etc. to
% Gerald Murray ---> murray@acm.org
%
% For tracking purposes - this is V3.0SP - JUNE 2007

\documentclass{acm_proc_article-sp}

\begin{document}

\title{Makumba: the Role of the Technology for the Sustainability of Amateur Programming Practice and Community}
\subtitle{the technological religion of a student tribe}
%
% You need the command \numberofauthors to handle the 'placement
% and alignment' of the authors beneath the title.
%
% For aesthetic reasons, we recommend 'three authors at a time'
% i.e. three 'name/affiliation blocks' be placed beneath the title.
%
% NOTE: You are NOT restricted in how many 'rows' of
% "name/affiliations" may appear. We just ask that you restrict
% the number of 'columns' to three.
%
% Because of the available 'opening page real-estate'
% we ask you to refrain from putting more than six authors
% (two rows with three columns) beneath the article title.
% More than six makes the first-page appear very cluttered indeed.
%
% Use the \alignauthor commands to handle the names
% and affiliations for an 'aesthetic maximum' of six authors.
% Add names, affiliations, addresses for
% the seventh etc. author(s) as the argument for the
% \additionalauthors command.
% These 'additional authors' will be output/set for you
% without further effort on your part as the last section in
% the body of your article BEFORE References or any Appendices.

\numberofauthors{2}
\author{
\alignauthor
Cristian Bogdan\\
       \affaddr{School of Computer Science and Communication}\\
       \affaddr{Royal Institute of Technology, 10044 Stockholm, Sweden}\\
       \email{cristi@csc.kth.se}
% 2nd. author
\alignauthor
Rudolf Mayer\\
       \affaddr{Institute of Software Technology and Interactive Systems}\\
       \affaddr{Vienna University of Technology}\\
       \affaddr{Favoritenstrasse 9--11, Vienna, Austria}\\
       \email{mayer@ifs.tuwien.ac.at}
}

\maketitle
\begin{abstract}
We address the issue of sustainability of practice, which we regard as crucial for the sustainability of the community at large. In the absence of material reward, sustaining a specialized activity such as programming is not a trivial matter especially when members move often in and out of the community. Our case is the group of voluntary, amateur programmers from a European-wide student organization. We present this setting as an Amateur Community and as a Community of Practice, and show how such framing help in understanding sustainability of practice. Although being totally voluntary and managing a large intranet, the group has been thriving for over ten years. To explain such high practice sustainability we examine the role of the technology framework used by the group, which was designed together with its members nine years ago. We then propose a more general framework for understanding practice sustainability in the context of amateur communities of practice. 
\end{abstract}

% A category with the (minimum) three required fields
\category{H.4}{Information Systems Applications}{Miscellaneous}
%A category including the fourth, optional field follows...
\category{D.2.8}{Software Engineering}{Metrics}[complexity measures, performance measures]

\keywords{Sustainability of practice, Case studies, Amateur Community, Community of Practice} % NOT required for Proceedings

\section{Introduction}\label{sec:introduction}
\cite{bogdan03}
paper objectives

\section{Methods}\label{sec:method}
\begin{itemize}
\item participant observation
\item reflective account
\item questionnaires
\end{itemize}

\section{Community of Practice}\label{sec:cop}
learning path

\section{Amateur Community}\label{sec:amateur}
\begin{itemize}
\item Stebins, pre-professional
\item Ham paper
\item thesis
\end{itemize}

\section{Sustainability of Practice in Amateur Communities}\label{sec:sust}
\begin{itemize}
\item why this is important, generalize to any community
\item quote sustainability in PD, to differentiate from environmental sustainability
\end{itemize}

\section{The Tech Committee}\label{sec:itc}
\begin{itemize}
\item as an amateur community of practice
\item karamba
\end{itemize}

\section{Makumba, the Tech Committee tool}\label{sec:makumba}
relevant aspects only... non-technical description

\section{Sustainability of the Tech Committee}
\begin{itemize}
\item table with karamba size, as it grew over years
\item table with number of members per year, at various makumba levels
\item some illustrative member trajectories (e.g. Gwen from little programming knowledge to Accenture, \item see chap 4 of thesis)
\item questionnaire commentary
\end{itemize}

\section{Discussion}\label{sec:disco}


\section{Conclusions}\label{sec:conclusions}

%ACKNOWLEDGMENTS are optional
\section{Acknowledgments}\label{sec:acknowledgments}
Thanks to the students, members and associates of the Tech Committee who have worked hard to overcome Makumba imperfections and to make the Intranet what it is today, while also taking time to answer our surveys.  Thanks are also due to all the non-BEST Makumba contributors.  Anonymous1 and Anonymous2 have supervised this work with good advice during the crucial Makumba design phases.

%
% The following two commands are all you need in the
% initial runs of your .tex file to
% produce the bibliography for the citations in your paper.
\bibliographystyle{abbrv}
\bibliography{cct2009-makumba} 

% For the final version:
% ACM needs 'a single self-contained file'!
%

\balancecolumns
% That's all folks!
\end{document}
