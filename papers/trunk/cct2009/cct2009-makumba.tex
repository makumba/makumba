% THIS IS SIGPROC-SP.TEX - VERSION 3.0
% WORKS WITH V3.1SP OF ACM_PROC_ARTICLE-SP.CLS
% JUNE 2007
%
% It is an example file showing how to use the 'acm_proc_article-sp.cls' V3.1SP
% LaTeX2e document class file for Conference Proceedings submissions.
% ----------------------------------------------------------------------------------------------------------------
% This .tex file (and associated .cls V3.1SP) *DOES NOT* produce:
%       1) The Permission Statement
%       2) The Conference (location) Info information
%       3) The Copyright Line with ACM data
%       4) Page numbering
% ---------------------------------------------------------------------------------------------------------------
% It is an example which *does* use the .bib file (from which the .bbl file
% is produced).
% REMEMBER HOWEVER: After having produced the .bbl file,
% and prior to final submission,
% you need to 'insert'  your .bbl file into your source .tex file so as to provide
% ONE 'self-contained' source file.
%
% For tracking purposes - this is V3.0SP - JUNE 2007

\documentclass{acm_proc_article-sp}

\usepackage{url}

\begin{document}

\title{Makumba: the Role of the Technology for the Sustainability of Amateur Programming Practice and Community}
\subtitle{the technological religion of a student tribe}

% \numberofauthors{2}
% \author{
% \alignauthor
% Cristian Bogdan\\
%        \affaddr{School of Computer Science and Communication}\\
%        \affaddr{Royal Institute of Technology, 10044 Stockholm, Sweden}\\
%        \email{cristi@csc.kth.se}
% \alignauthor
% Rudolf Mayer\\
%        \affaddr{Institute of Software Technology and Interactive Systems}\\
%        \affaddr{Vienna University of Technology}\\
%        \affaddr{Favoritenstrasse 9--11, Vienna, Austria}\\
%        \email{mayer@ifs.tuwien.ac.at}
% }

% empty authors block for double-blind review
\numberofauthors{1}
\author{
\alignauthor
\hspace{10mm} \\
       \affaddr{\hspace{10mm} }\\
       \affaddr{\hspace{10mm} }\\
       \affaddr{\hspace{10mm} }\\
       \email{\hspace{10mm} }
}

\maketitle
\begin{abstract}
We address the issue of sustainability of practice, which we regard as crucial for the sustainability of the community at large. In the absence of material reward, sustaining a specialized activity such as programming is not a trivial matter especially when members move often in and out of the community. Our case is the group of voluntary, amateur student programmers from a European-wide student organization. We present this setting as an Amateur Community and as a Community of Practice, and show how such framing helps in understanding sustainability of practice. Although being totally voluntary and managing a large intranet, the group has been thriving for over ten years. To explain such high practice sustainability we examine the role of the technology framework used by the group, which was designed together with its members nine years ago. We then propose a more general framework for understanding practice sustainability in the context of amateur communities of practice. 
\end{abstract}

% A category with the (minimum) three required fields
\category{H.4}{Information Systems Applications}{Miscellaneous}
%A category including the fourth, optional field follows...
\category{D.2.8}{Software Engineering}{Metrics}[complexity measures, performance measures]

\keywords{Sustainability of practice, Case studies, Amateur Community, Community of Practice} % NOT required for Proceedings

\section{Introduction}\label{sec:introduction}
%TODO more precise numbers
The Intranet of an European-wide organization had just over 600 dynamic pages at its launch in 2002. The Intranet grew steadily in both size and functionality
%TODO functionality examples
 and at the end of 2008 it has over 3000 dynamic pages. Throughout, except for a few glitches, the system was highly available and fast. Today the Intranet has around 2000 users from among the organization members, and its public part can be accessed by 4000-5000 organization customers at a time.

Such figures would not be surprising for a professional international organization, which hires or outsources professional programmers and system administrators. 
The figures are however unusual for a voluntary student organization, whose IT crew (referred to as the Tech Committee) is made of amateur voluntary student programmers who often do not study Computer Science or related subjects, or are at the early stages of their education.  
Since 2002  the Tech Committee had over 10 active members at any given time, and the Intranet was extended with more and more subsystems even if about one third of the Tech Committee membership was renewed each year. Reflection on this {\it sustainability} is the subject matter of this paper. We further assert that the design of the technology used to build the Intranet, called Makumba, is an essential ingredient of the Tech Committee sustainability.
Prior to Makumba introduction, the student organization had disparate IT systems, and the Tech Committee was formed in 1997 to maintain them but had much lower membership counts and was highly dependent on a few people.

%TODO: "either young or dead", the problem of sustainability
In this paper we aim to characterize {\it sustainability of practice and community} based on this positive experience. We also aim to discuss the role of technology in achieving such sustainability, and to draw design implications that would help to devise technologies that encourage sustainability. 

While we recognize the importance of {\it environmental sustainability}  in IT and interaction design \cite{blevis07} we focus on a topic that is only abstractly related, yet we believe important in community and technology research. It is e.g. common knowledge that most virtual communities are "either young or dead" \cite{pargman05}, and various cookbooks for making the community thrive were produced \cite{goodwin94}. Even for non-virtual communities such as our student organization and its Tech Committee, it is sometimes difficult to sustain a specialized activity within the community, and this issue was raised under the name of sustainability in the field of Participatory Design (PD), e.g. as a basic principle of a PD method \cite{kensing98} or as a call for "self-sustaining [PD] processes within work settings" \cite{clement93}.  While managerial aspects such as leadership and incentives for work are certainly important in sustaining an activity, we also advocate a focus on the methods and tools used in such specialized activities, as their being easy to learn or entertaining to use should lead to better sustainability of the respective practice. As immediately apparent liabilities to practice sustainability one can exemplify lack of material rewards, or frequent changes in community membership, both of which affect voluntary student organizations. A related useful notion is {\it self-sustainabilty}, whereby a specialized activity such as PD \cite{clement93} or programming is started by professional intervention in the setting, yet it is able to continue using only community resources (human and otherwise), after the specialized professionals leave the setting. In our case both types of activity were initiated in the student organization  by the first author as a Human-Compute Interaction researcher and computer engineer, who left the setting in 2003. The programming activity and its sustainability are addressed in this paper.

In what follows, we will frame the Tech Committee as a Community of Practice \cite{lave_wenger91, wenger98} and as an Amateur Community \cite{bogdan03, bogdan_bowers07}. We will then introduce the topic of practice sustainability, highly related to learning and essential for community sustainability. We present the student organization Intranet and the Makumba technology that powers it, then we describe in detail the sustainable evolution of the Intranet and the Tech Committee, as it comes out from examining the source code repositories and Tech Committee membership lists. We then present the results of several surveys 
%TODO: maybe just one survey?
that we used to elicit data on Makumba learning aspects in general and sustainability especially. We then discuss our results with a focus on learning, sustainability and design implications for technology.

%TODO maybe mention the dialog between authors if we manage to have it

\section{Setting}\label{sec:setting}
%TODO Rudi please fix the numbers
The student organization running the Intranet of our interest was grounded in 1988 and is currently present in NN mainly technical universities across Europe (grew from NN1 in 2003), and currently has about MMM members. Their aim is to organize complementary education in the form of courses and competitions for students of the member universities (i.e. not just for its members), to provide them with leisure events and to assist them in finding jobs, etc. 
% TODO please add aims/details
The organization maintains contact with and raises funds from the European Union bodies and from a number of company partners. Internally the organization has two statutory meetings per year and several less formal, smaller meetings in between. While most members only worked on international topics during such meetings, and for the rest they worked in their local organization chapters, since 1997 the organization was able to sustain "committees" on several topics (marketing, fund-raising, IT- the Tech Committee, complementary education program management, etc). The committees met in the international meetings but kept on working on their topic also in between meetings, employing e-mail and instant messaging systems, as well as dedicated tools as part of the Intranet.

The Intranet is supporting the activities of the student organization. It was assembled as an integrated system in 2002 from several systems: 
\begin{itemize}
\item an event application system for complementary education courses. The system registered student applications to the course events, and let the organization manage their acceptance (to one event out of maximum three applied for) and participation. The system had been re-built each year since 1993 to different levels of completeness, using various technologies, and finally 'stabilized' as a Java system which was  re-used for several annual editions of the course program starting in 1997. This system was helping the management of the most important student organization activity and was heavily dependent on the first author until its Makumba re-implementation in 2002.
\item an internal document archive and member profile management system known as the "Private Area". The system initially functioned as manually maintained WWW pages, and it was automated using Lotus Notes in 1998, when internal event management was added to manage applications and participations in statutory meetings and other internal events. At the time, a need for a common technology for the IT systems of the organization was recognized and Lotus Notes was intended to be that technology. However its shortcomings for the Tech Committee context were recognized and Makumba was designed in response.
\item a "virtual jobfair" allowing companies to post job adverts, and students to post their profiles. This was a Java-based system launched in 2000.
% TODO, I think it was 2000, not sure how important this is, it was mostly unused for many years
\end{itemize}
Besides heavily extending and integrating the above subsystems, several Intranet features were added since its inception: a Wiki, a training database, a unique sign-in system that allowed students to share their accounts between the course application system and the "virtual jobfair", etc.
%TODO more here, it feels like little was done since 2002 :)

The Tech Committee is in charge of developing and maintaining the Intranet, as well as supporting it with activities such as helpdesk. The committee also coordinates activities of interaction design for further new areas of the Intranet. There are two formal membership levels ("trainee" and "active member") and there is a formal coordinator, elected each year, who is also a member of the student organization overall coordination bodies. In its early (less sustainable) days the Tech Committee consisted of 1-5 members who were mainly responsible for the respective systems and had difficulties backing each other up when they did not have time for voluntary commitment. Membership levels increased after 2002, and new members are usually attracted in international meetings with a three-hour Makumba training.
% TODO more about the training, or after it: are they given Intranet tasks? Do they do Intranet-based exercises to change some interface or browse some Intranet data?

\section{Methods}\label{sec:method}
We regard sustainability as a long-term matter that cannot be investigated with a short focused study. Throughout our contact with and involvement in the Tech Committee we were concerned with sustainability and at times we considered  theoretical frameworks and recipes for how to achieve it, thus sustainability was an ever-present research issue for us, but also a practical issue in the community life. Both authors were active in the Tech Committee at different times (1997-2003 and 2004-2008 respectively), and {\it participant observation} was a conscious investigation approach for the first author, who also designed the Makumba technology together with Tech Committee members in a conscious act of {\it cooperative design} \cite{greenbaum_kyng91} . This paper is the occasion for a {\it reflection exercise} \cite{schon83}, on the part of the second author. During our membership we had access to the e-mail traffic and other communication of the Tech Committee. 

We are highly aware that our first-hand involvement with the Tech Committee is a potential hinder for us producing an 'objective' account of its sustainability and on other matters of interest, and this is probably not uncommon in community-related research endeavors. We therefore complemented our personal experiences with both non-elicited and elicited data, as described below. Furthermore, our involvement having taken place {\it at different times} has resulted in a fruitful confrontation process that allowed us to depart from our personal opinions and arrive at more reliable and valid results.
%TODO more about dialog between authors

For assessing quantitatively and qualitatively the sustainability of the Tech Committee and its practice, we regard as non-elicited data the {\it code repository} that the community has maintained, which allowed us to assess and reflect upon the progress of the Tech Committee work. The repository uses the Concurrent Versioning Systems (CVS) technology, which allows us to see the incremental changes and additions that were made to the code, their time, and their authors. We are thus able to reconstitute the Intranet code as it was at any moment in time since 2003. A number of code analysis tools are also available for investigating CVS repositories, and we found them useful for our inquiry.

We have also elicited data from the Tech Committee members, for purposes related to the Makumba technology. In 2002 the first author evaluated the design of Makumba with a questionnaire that had 12 respondents. In 2005 the first author also ran a questionnaire
%TODO number of respondents
to asses the status of Tech Committee work and technology use, and thus indirectly look at its sustainability. Finally, both authors designed a questionnaire at the end of 2008 where 30 out of the 45 past (since 2003) and present members who were approached reflected on their activity in the Tech Committee over their whole 3-4 year-long membership period. This questionnaire provides our paper with both quantitative and qualitative data. 

A further form of non-elicited data is constituted by Tech Committee {\it membership lists} at different times during the committee activity since 2002. Such lists can be made by examining member profiles in the Intranet. Number of members, as well as their level of activity are useful indicators in assessing sustainability. Therefore the membership lists were complemented with levels of member activity as elicited from committee leaders and self-assessed by members themselves in questionnaires. Personal acquaintance with many of the members and knowledge of their skill evolution has come to complement this further. Some members continued on an IT career after graduation, and this was yet another indicator of their Intranet development and generic IT skills.

\section{Community of Practice}\label{sec:cop}
We found the Community of Practice 
learning path, software-related communities of practice

\section{Amateur Community}\label{sec:amateur}
\begin{itemize}
\item Stebins, pre-professional
\item Ham paper
\item thesis
\end{itemize}

\section{Sustainability of Practice in Amateur Communities}\label{sec:sust}
\begin{itemize}
\item Goodwin nine principles for making virtual communities work
\item why this is important, generalize to any community
\item quote sustainability in PD, to differentiate from environmental sustainability
\end{itemize}

\section{Makumba, the Tech Committee tool}\label{sec:makumba}
relevant aspects only... non-technical description

\section{Sustainability of the Tech Committee}
\begin{itemize}
\item Tech Committee as an amateur community of practice
\item table with karamba size, as it grew over years
\item table with number of members per year, at various makumba levels
\item some illustrative member trajectories (e.g. Gwen from little programming knowledge to Accenture, see chap 4 of thesis)
\item questionnaire commentary
\end{itemize}

\begin{table}\label{tab:karamba}
	\centering
	\caption{Size of 'Karamba' web application}
	\begin{tabular}{c|r|r}
		\hline
		\hline
		BEST Year 		& Number of files 	& Lines of Code\\
		\hline
		\hline
		2002/2003		& 850				& 55,000 \\
		\hline
		2003/2004		& 1,200				& 75,000 \\
		\hline
		2004/2005		& 1,700				& 85,000 \\
		\hline
		2005/2006		& 2,300				& 132,000 \\
		\hline
		2006/2007		& 3,050				& 200,000 \\
		\hline
		2007/2008		& 3,550				& 305,000 \\
		\hline
		Dec. 2008		& 3,750				& 312,000 \\
		\hline
		\hline
	\end{tabular}
\end{table} 

% TODO: exclude http://private.best.eu.org/itc/cvs-complete/2006-09.html#223, parts of http://private.best.eu.org/itc/cvs-complete/2006-09.html#224

\begin{table}\label{tab:itd-members}
	\centering
	\caption{Members in the Tech Committee}
	\begin{tabular}{c|r|r}
		\hline
		\hline
		BEST Year 		& Very active members 	& Total active members 	\\
		\hline
		\hline
		2002/2003		& 						&  \\
		\hline
		2003/2004		& 						&  \\
		\hline
		2004/2005		& 						&  \\
		\hline
		2005/2006		& 						&  \\
		\hline
		2006/2007		& 						&  \\
		\hline
		2007/2008		& 						&  \\
		\hline
		Dec. 2008		& 						&  \\
		\hline
		\hline
	\end{tabular}
\end{table} 

\section{Discussion}\label{sec:disco}
dialogue cristi-rudi... plus importance of leadership in sustainability... plus importance of the initial importer...


\section{Conclusions}\label{sec:conclusions}

%ACKNOWLEDGMENTS are optional
\section{Acknowledgments}\label{sec:acknowledgments}
Thanks to the students, members and associates of the Tech Committee who have worked hard to overcome Makumba imperfections and to make the Intranet what it is today, while also taking time to answer our surveys.  Thanks are also due to all the non-BEST Makumba contributors.  Anonymous1 and Anonymous2 have supervised this work with good advice during the crucial Makumba design phases.

%
% The following two commands are all you need in the
% initial runs of your .tex file to
% produce the bibliography for the citations in your paper.
\bibliographystyle{abbrv}
\bibliography{cct2009-makumba} 

% For the final version:
% ACM needs 'a single self-contained file'!
%

\balancecolumns
% That's all folks!
\end{document}
