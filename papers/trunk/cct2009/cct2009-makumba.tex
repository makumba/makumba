% THIS IS SIGPROC-SP.TEX - VERSION 3.0
% WORKS WITH V3.1SP OF ACM_PROC_ARTICLE-SP.CLS
% JUNE 2007
%
% It is an example file showing how to use the 'acm_proc_article-sp.cls' V3.1SP
% LaTeX2e document class file for Conference Proceedings submissions.
% ----------------------------------------------------------------------------------------------------------------
% This .tex file (and associated .cls V3.1SP) *DOES NOT* produce:
%       1) The Permission Statement
%       2) The Conference (location) Info information
%       3) The Copyright Line with ACM data
%       4) Page numbering
% ---------------------------------------------------------------------------------------------------------------
% It is an example which *does* use the .bib file (from which the .bbl file
% is produced).
% REMEMBER HOWEVER: After having produced the .bbl file,
% and prior to final submission,
% you need to 'insert'  your .bbl file into your source .tex file so as to provide
% ONE 'self-contained' source file.
%
% For tracking purposes - this is V3.0SP - JUNE 2007

\documentclass{acm_proc_article-sp}

\begin{document}

\title{Makumba: the Role of the Technology for the Sustainability of Amateur Programming Practice and Community}
\subtitle{the technological religion of a student tribe}

\numberofauthors{2}
\author{
\alignauthor
Cristian Bogdan\\
       \affaddr{School of Computer Science and Communication}\\
       \affaddr{Royal Institute of Technology, 10044 Stockholm, Sweden}\\
       \email{cristi@csc.kth.se}
% 2nd. author
\alignauthor
Rudolf Mayer\\
       \affaddr{Institute of Software Technology and Interactive Systems}\\
       \affaddr{Vienna University of Technology}\\
       \affaddr{Favoritenstrasse 9--11, Vienna, Austria}\\
       \email{mayer@ifs.tuwien.ac.at}
}

\maketitle
\begin{abstract}
We address the issue of sustainability of practice, which we regard as crucial for the sustainability of the community at large. In the absence of material reward, sustaining a specialized activity such as programming is not a trivial matter especially when members move often in and out of the community. Our case is the group of voluntary, amateur student programmers from a European-wide student organization. We present this setting as an Amateur Community and as a Community of Practice, and show how such framing help in understanding sustainability of practice. Although being totally voluntary and managing a large intranet, the group has been thriving for over ten years. To explain such high practice sustainability we examine the role of the technology framework used by the group, which was designed together with its members nine years ago. We then propose a more general framework for understanding practice sustainability in the context of amateur communities of practice. 
\end{abstract}

% A category with the (minimum) three required fields
\category{H.4}{Information Systems Applications}{Miscellaneous}
%A category including the fourth, optional field follows...
\category{D.2.8}{Software Engineering}{Metrics}[complexity measures, performance measures]

\keywords{Sustainability of practice, Case studies, Amateur Community, Community of Practice} % NOT required for Proceedings

\section{Introduction}\label{sec:introduction}
%TODO more precise numbers
The Intranet of an European-wide organization is developed and maintained since 1995. In 2002 when the Intranet components were ported to a common technology, the Intranet had just over 600 dynamic pages. The Intranet grew steadily in both size and functionality
%TODO functionality examples
 and at the end of 2008 it has over 3000 dynamic pages. Throughout, except for a few glitches, the system was highly available and fast. Today the Intranet has around 2000 users from among the organization members, and its public part can be accessed by 4000-5000 organization customers at a time.

Such figures would not be surprising for a professional international organization, which hires or outsources professional programmers and system administrators. The figures are however unusual for a voluntary student organization, whose IT crew (referred to as the Tech Committee) is made of amateur voluntary student programmers who often do not study Computer Science or related subjects, or are at the early stages of their education.  In its incipient stages the Tech Committee consisted of 1-5 members and always seemed to have an uncertain future, as its members graduated or simply lacked the time for voluntary programming or Intranet maintenance, and there were few other members who could back them up. Difficulty to back up or continue a member's work was even more pronounced as the Tech Committee managed a number of applications that had been spontaneously started by some volunteer, using various technologies that they happened to master. The group was, in the terms introduced by this paper, not {\it sustainable}.  Since 2002 when all the Intranet components were ported to the same technology (called Makumba), designed with smooth learning and sustainability in mind, the Tech Committee had over 10 active members at any given time, and the Intranet could grow and develop even if about one third of the membership was renewed each year. Reflection on this sustainability is the subject matter of this paper.

We therefore aim to characterize sustainability of practice and community based on this positive experience. We also aim to discuss the role of technology in achieving such sustainability, and to draw design implications that would help to devise technologies that encourage sustainability. 

In what follows, we will frame the Tech Committee as a Community of Practice \cite{lave_wenger91, wenger98} and as an Amateur Community \cite{bogdan03, bogdan_bowers07}. We will then introduce the topic of practice sustainability, highly related to learning and essential for community sustainability. We present the student organization Intranet and the Makumba technology that powers it, then we describe in detail the sustainable evolution of the Intranet and the Tech Committee, as it comes out from examining the source code repositories and Tech Committee membership lists. We then present the results of several surveys 
%TODO: maybe just one survey?
that we used to elicit data on Makumba learning aspects in general and sustainability especially. We then discuss our results with a focus on learning, sustainability and design implications for technology.

%TODO maybe mention the dialog between authors if we manage to have it

\section{Methods}\label{sec:method}
\begin{itemize}
\item participant observation
\item reflective account
\item questionnaires
\end{itemize}

\section{Community of Practice}\label{sec:cop}
learning path, software-related communities of practice

\section{Amateur Community}\label{sec:amateur}
\begin{itemize}
\item Stebins, pre-professional
\item Ham paper
\item thesis
\end{itemize}

\section{Sustainability of Practice in Amateur Communities}\label{sec:sust}
\begin{itemize}
\item why this is important, generalize to any community
\item quote sustainability in PD, to differentiate from environmental sustainability
\end{itemize}

\section{The Tech Committee}\label{sec:itc}
\begin{itemize}
\item as an amateur community of practice
\item karamba
\end{itemize}

\section{Makumba, the Tech Committee tool}\label{sec:makumba}
relevant aspects only... non-technical description

\section{Sustainability of the Tech Committee}
\begin{itemize}

\begin{table}\label{tab:karamba}
	\centering
	\caption{Size of 'Karamba' web application}
	\begin{tabular}{c|r|r}
		\hline
		\hline
		BEST Year 		& Number of files 	& Lines of Code\\
		\hline
		\hline
		2002/2003		& 850				& 55,000 \\
		\hline
		2003/2004		& 1,200				& 75,000 \\
		\hline
		2004/2005		& 1,700				& 85,000 \\
		\hline
		2005/2006		& 2,300				& 132,000 \\
		\hline
		2006/2007		& 3,050				& 200,000 \\
		\hline
		2007/2008		& 3,550				& 305,000 \\
		\hline
		Dec. 2008		& 3,750				& 312,000 \\
		\hline
		\hline
	\end{tabular}
\end{table} 

% TODO: exclude http://private.best.eu.org/itc/cvs-complete/2006-09.html#223, parts of http://private.best.eu.org/itc/cvs-complete/2006-09.html#224

\begin{table}\label{tab:itd-members}
	\centering
	\caption{Members in the Tech Committee}
	\begin{tabular}{c|r|r}
		\hline
		\hline
		BEST Year 		& Very active members 	& Total active members 	\\
		\hline
		\hline
		2002/2003		& 						&  \\
		\hline
		2003/2004		& 						&  \\
		\hline
		2004/2005		& 						&  \\
		\hline
		2005/2006		& 						&  \\
		\hline
		2006/2007		& 						&  \\
		\hline
		2007/2008		& 						&  \\
		\hline
		Dec. 2008		& 						&  \\
		\hline
		\hline
	\end{tabular}
\end{table} 

\item table with karamba size, as it grew over years
\item table with number of members per year, at various makumba levels
\item some illustrative member trajectories (e.g. Gwen from little programming knowledge to Accenture, see chap 4 of thesis)
\item questionnaire commentary
\end{itemize}

\section{Discussion}\label{sec:disco}
dialogue cristi-rudi... plus importance of leadership in sustainability... plus importance of the initial importer...


\section{Conclusions}\label{sec:conclusions}

%ACKNOWLEDGMENTS are optional
\section{Acknowledgments}\label{sec:acknowledgments}
Thanks to the students, members and associates of the Tech Committee who have worked hard to overcome Makumba imperfections and to make the Intranet what it is today, while also taking time to answer our surveys.  Thanks are also due to all the non-BEST Makumba contributors.  Anonymous1 and Anonymous2 have supervised this work with good advice during the crucial Makumba design phases.

%
% The following two commands are all you need in the
% initial runs of your .tex file to
% produce the bibliography for the citations in your paper.
\bibliographystyle{abbrv}
\bibliography{cct2009-makumba} 

% For the final version:
% ACM needs 'a single self-contained file'!
%

\balancecolumns
% That's all folks!
\end{document}
